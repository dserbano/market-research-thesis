\chapter{Conclusions}

\textsl{This last chapter reports the evaluations and considerations regarding the completion of the set objectives and offers a summary and reflection on the research.}
\\\\
This project attempted to implement a prototype of a general purpose \ac{MRS} that gathers data from various sources and generates an array of marketing insights. The implemented prototype is divided into a Front-End application and into a Back-End server that run into Docker containers. The implemented \ac{MRS} is currently able to gather some data using some web scrapers and can analyse this data to produce forecasts and clusters for products. The proposed \ac{MRS} can be optimized in various ways. For example, APIs can be used instead of web scrapers and the data sources can be brought locally to the server. Marketing data is non stationary and it is difficult to model due to data points having different distributions. This thesis shows a way to model such data and perform accurate forecasting using kernel machines. A \ac{KRLS}-T model was used that is able to more accurately predict what comes next by keeping a set of distributions for each data point. It can be shown that the choice of the kernel greatly affects the performance of kernel machines. In non-stationary scenarios, this online algorithm implements a forgetting mechanism that can track the changes of the observed model by weighting past data less heavily than more recent data. The forgetting mechanism improves performance by generating a curve that does not fit so well the training data but can provide very good forecasts. The dictionary size is also a factor as more training data can empower the algorithm to also perform better. The \ac{KRLS}-T algorithm achieves the lowest MSE when using a rational quadratic kernel, no forgetting factor and a large dictionary size. However, the forecasts make more sense when using also a forgetting factor. When using a radial basis function the performance is inferior. 




